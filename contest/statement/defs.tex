%\documentclass[a4paper,11pt]{article}
\usepackage{amsmath}
\usepackage{amsfonts}
\usepackage[brazil]{babel}
\usepackage[utf8]{inputenc}
\usepackage[T1]{fontenc}
\usepackage{indentfirst}
\usepackage{float}
\usepackage{fancyvrb}
\usepackage{pdftexcmds}
\usepackage{multicol}
\usepackage{amsmath}
\usepackage{hyperref}
\usepackage{ifthen}
\usepackage{listings}
\usepackage{epsfig}
\usepackage{tikz}

\setlength{\marginparwidth}{0pt}
\setlength{\oddsidemargin}{-0.25cm}
\setlength{\evensidemargin}{-0.25cm}
\setlength{\marginparsep}{0pt}

\setlength{\parindent}{0cm}
\setlength{\parskip}{5pt}

\setlength{\textwidth}{16.5cm}
\setlength{\textheight}{25.5cm}

\setlength{\voffset}{-1in}

\newcommand{\insereArquivo}[1]{
	\ifnum0\pdffilesize{#1}>0
		\VerbatimInput[xleftmargin=0mm,numbers=none,obeytabs=true]{#1}\vspace{.5em}
	\fi
}

%\newcommand{\textoDiversasInstancias}{A entrada é composta por
%diversos casos de teste. }
\newcommand{\textoDiversasInstancias}{}

%\newcommand{\arquivoProblema}[1]{\vspace{-0.3cm} \noindent {\em
%Arquivo: \texttt{#1.[c|cpp|java]} \\}}

\newcommand{\textoSaidaPadrao}{\vspace{0.2cm} \noindent \emph{A
		saída deve ser escrita na saída padrăo. }}

\newcommand{\textoEntradaPadrao}{\vspace{0.2cm} \noindent \emph{A
		entrada deve ser lida da entrada padrăo. }}

% Language definitions (choose portuguese/english)
%\def\lang{english}
\def\lang{portuguese}

\def\english{english}
\ifx\lang\english
	\def\strExplanation{Explanation of sample }
	\def\strInput{Input}
	\def\strInputdescline{The input consists of a single line that contains }
	\def\strOutput{Output}
	\def\strOutputdescline{Output a single line with }
	\def\strOutputfloatnoticeA{The result must be output as a rational number with exactly }
	\def\strOutputfloatnoticeB{digits after the decimal point, rounded if necessary.}
	\def\strOutputAbsRelNotice{The output must have an absolute or relative error of at most }
	\def\strSampleinput{Sample input }
	\def\strSampleoutput{Sample output }
\else
	\def\strExplanation{Explicação do exemplo }
	\def\strInput{Entrada}
	\def\strInputdescline{A entrada consiste de uma única linha que contém }
	\def\strOutput{Saída}
	\def\strOutputdescline{Seu programa deve produzir uma única linha com }
	\def\strOutputfloatnoticeA{O resultado deve ser escrito como um número racional com exatamente }
	\def\strOutputfloatnoticeB{dígitos após o ponto decimal, arredondado se necessário.}
	\def\strOutputAbsRelNotice{O resultado deve ter um erro absoluto ou relativo de no máximo }
	\def\strSampleinput{Exemplo de entrada }
	\def\strSampleoutput{Exemplo de saída }
\fi


\newcommand{\inputdesc}[1]{
	\subsection*{\strInput}
	{#1} }

\newcommand{\inputdescline}[1]{
	\subsection*{\strInput}
	\strInputdescline {#1} }

\newcommand{\outputdesc}[1]{
	\subsection*{\strOutput}
	{#1}}

\newcommand{\sampledesc}[2]{
	\subsection*{\strExplanation #1}
	{#2}}


\newcommand{\incat}[1]{sample-#1.in}
\newcommand{\solcat}[1]{sample-#1.sol}
\newcounter{problemcounter}
\setcounter{problemcounter}{0}
\newcounter{exemplocounter}[problemcounter]
\setcounter{exemplocounter}{0}

\newcommand{\exemplo}[2]{
	{\small

		\stepcounter{exemplocounter}

		\begin{minipage}[c]{0.9\textwidth}
			\begin{center}
				\begin{tabular}{|l|l|} \hline
					\begin{minipage}[t]{0.5\textwidth}
						\bf{Exemplo de entrada \arabic{exemplocounter}}
						\insereArquivo{#1}
					\end{minipage}
					 &
					\begin{minipage}[t]{0.5\textwidth}
						\bf{Exemplo de saída \arabic{exemplocounter}}
						\insereArquivo{#2}
					\end{minipage} \\
					\hline
				\end{tabular}
			\end{center}
		\end{minipage} % leave next line empty

	}
} % exemplo

\newcommand{\explanation}[1]{
	\textbf{\strExplanation\theexemplocounter}

	{#1} % leave empty line

} % explanation

%\newcommand{\incluir}[1]{
%\input{#1}
%}
